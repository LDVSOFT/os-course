\begin{frame}
\frametitle{Честное планирование}

Предыдущие подходы к планированию пытались:
\begin{itemize}
  \item оптимизировать время ожидания или уменьшить простаивание (SJF);
  \item предоставить гарантии на время отклика (RR);
  \item подобрать оптимальный "приоритет" для задачи (MLFQ);
\end{itemize}

\onslide<2->{Теперь посмотрим на проблему с другой стороны:
\begin{itemize}
  \item давайте честно делить ресурсы CPU между потоками (процессами,
        пользователями);
\end{itemize}}
\end{frame}

\begin{frame}
\frametitle{Честное планирование}

Честное планирование не совсем очевидная вещь:
\begin{itemize}
  \item потоки могут делать IO в произвольные моменты времени
        \begin{itemize}
          \item нельзя сказать потоку работать какое-то конкретное время;
          \item время работы можно ограничить только сверху;
        \end{itemize}
  \item могут появляться новые потоки
        \begin{itemize}
          \item не очевидно как должна работать "честность" при изменениях
                условий;
        \end{itemize}
  \item<2-> если вы не знаете что делать - используйте рандом;
\end{itemize}
\end{frame}

\begin{frame}
\frametitle{Лотерейное планирование}

У каждого потока (процесса или пользователя) есть набор билетов
\begin{itemize}
  \item при распределении ресурсов выбирается случайный номер билета
        \begin{itemize}
          \item ресурс получает владелец выигравшего билета;
          \item вероятность выигрыша определяется количеством билетов;
        \end{itemize}
  \item свои билеты можно передавать/разделять:
        \begin{itemize}
          \item процесс может разделить свои билеты между потоками;
          \item пользователь может делить свои билеты между процессами;
          \item если поток ждет другой поток, то он может передать ему свои
                билеты;
        \end{itemize}
  \item можно создавать новые билеты или уничтожать старые.
\end{itemize}
\end{frame}

\begin{frame}
\frametitle{Completely Fair Scheduler}

История планировщиков Linux:
\begin{enumerate}
  \item Linux 1.2 - Round Robin;
  \item Linux 2.2 - поддержка SMP (многопроцессорное планирование);
  \item Linux 2.4 - добавили планировщик со сложностью $O(N)$ (он обходил все
        потоки в системе);
  \item Linux 2.6 - O(1) scheduler (это название), вариация на тему MLFQ;
  \item Linux 2.6 - Completely Fair Scheduler, "честное" разделение CPU на
        основе динамических приоритетов без рандомизации;
\end{enumerate}
\end{frame}

\begin{frame}
\frametitle{Completely Fair Scheduler}

\begin{itemize}
  \item<1-> у каждого потока есть virtual runtime, чем меньше virtual runtime
        тем выше приоритет:
        \begin{itemize}
          \item \emph{активные} потоки хранятся в красно-черном дереве
                упорядоченными по virtual runtime;
          \item virtual runtime изменяется, когда поток работает (т. е.,
                упрощенно, изменение virtual runtime = k * CPU time);
        \end{itemize}
  \item<2-> какой virtual runtime взять для нового (или разблокированного)
        потока? Берем наибольшее значение из:
        \begin{itemize}
          \item минимальный virtual runtime в дереве;
          \item virtual runtime потока (если это не новый поток);
        \end{itemize}
\end{itemize}
\end{frame}
