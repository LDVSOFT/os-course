\begin{frame}
\frametitle{Поток исполнения и контекст потока}
Поток (thread, поток исполнения) - набор исполняемых инструкций процессора
\begin{itemize}
  \item указатель команд (instruction pointer, IP, RIP в x86-64) - указывает на
        текущую инструкцию;
  \item регистр флагов (flag register, state register, RFLAGS в x86-64) -
        определяет поведение команд;
  \item указатель стека (stack pointer, RSP в x86-64) - указывает на вершину
        стека (указатель стека часто относят к регистрам общего назначения);
  \item прочие регистры общего назначения (и не только);
  \item процесс - поток связан с процессом (работает в рамках какого-то
        процесса).
\end{itemize}
\end{frame}

\begin{frame}
\frametitle{Переключение контекста}

\begin{itemize}
  \item сохраняем контекст одного потока в известное место (например, на стек
        этого потока);
  \item восстанавливаем контекст другого потока из известного места;
  \item если нужно переключаем процессы (например, меняем таблицу страниц);
  \item x86-32 поддерживает аппаратное переключение контекста\onslide<2->{,
        но никто им не пользовался, потому в x86-64 такой возможности больше
        нет.}
\end{itemize}
\end{frame}

\begin{frame}[fragile]
\frametitle{Пример переключения контекста для x86-64}

\begin{lstlisting}
; void switch_threads(void **old_sp, void *new_sp); // C signature

switch_threads:
        pushq %rbp        ; save volatile registers
        pushq %rbx
        pushq %r12
        pushq %r13
        pushq %r14
        pushq %r15

        movq %rsp, (%rdi) ; save SP
        movq %rsi, %rsp   ; restore SP

        popq %r15         ; restore volatile reigster
        popq %r14
        popq %r13
        popq %r12
        popq %rbx
        popq %rbp

        ret
\end{lstlisting}
\end{frame}

\begin{frame}
\frametitle{Когда переключаться между потоками?}

\begin{itemize}
  \item когда поток сам добровольно отдаст управление (невытесняющая
        многозадачность)
        \begin{itemize}
          \item добровольно вызовет перепланирование;
          \item совершит блокирующую операцию (mutex lock, condtion wait);
        \end{itemize}
  \item в обработчике прерывания (вытесняющая многозадачность)
        \begin{itemize}
          \item например, по прерыванию таймера;
        \end{itemize}
\end{itemize}
\end{frame}
