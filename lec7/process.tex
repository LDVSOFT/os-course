\begin{frame}
\frametitle{Создание процессов}

\begin{itemize}
  \item<1-> Способ создания процессов зависит от ОС:
    \begin{itemize}
      \item Windows использует CreateProcess (есть еще NtCreateProcess и ZwCreateProcess);
      \item UNIX-like ОС используют fork (мы тоже будем использовать его):
    \end{itemize}
  \item<2-> fork - создает почти точную копию процесса вызвавшего fork
    \begin{itemize}
      \item идентичная память (у каждого своя копия - не общая);
      \item файловые дескрипторы;
      \item процессы образуют дерево.
    \end{itemize}
\end{itemize}
\end{frame}

\begin{frame}[fragile]
\frametitle{Создание процессов}
\lstinputlisting[language=C++]{../ipc/fork.c}
\end{frame}

\begin{frame}
\frametitle{Завершение процессов}

\begin{itemize}
  \item<1-> В завершении процесса, как и в создании, участвуют двое:
    \begin{itemize}
      \item завершающийся процесс, вызывает exit или \_exit;
      \item процесс дожидающийся завершения вызывает wait или waitpid.
    \end{itemize}
  \item<2-> Чтобы удалить процесс из системы родитель должен вызывать wait:
    \begin{itemize}
      \item до тех пор, пока не вызван wait, завершившийся процесс находится в состоянии "zombie";
      \item если родитель умер раньше ребенка, то ребенка усыновит/удочерит другой процесс системы (например, init);
    \end{itemize}
\end{itemize}
\end{frame}

\begin{frame}[fragile]
\frametitle{Завершение процессов}
\lstinputlisting[language=C++]{../ipc/wait.c}
\end{frame}
