\begin{frame}[fragile]
\frametitle{Test-And-Set Lock}

Атомарные операции позволяют реализовать взаимное исключение гораздо проще чем
алгоритмы Петтерсона и Лэмпорта (и другие):

\begin{lstlisting}
#define LOCK_FREE 0
#define LOCK_HELD 1

struct spinlock {
    int state;
};

void lock(struct spinlock *lock)
{
    while (cmpxchg(&lock->state, LOCK_FREE, LOCK_HELD) != LOCK_FREE);
}

void unlock(struct spinlock *lock)
{
    xchg(&lock->state, LOCK_FREE);
}
\end{lstlisting}
\end{frame}

\begin{frame}
\frametitle{Test-And-Set Lock}

Test-And-Set Lock в такой реализации обладает рядом недостатков:
\begin{itemize}
  \item<2-> все потоки в цикле пытаются выполнить модификацию обще переменной:
        \begin{itemize}
          \item вспомните MESI - писать может только один;
          \item много бесполезного трафика по системной шине;
        \end{itemize}
  \item<3-> в такой реализации мы не можем давать никаких гарантий честности;
  \item<4-> она не работает (приводит к deadlock-у).
\end{itemize}
\end{frame}

\begin{frame}
\frametitle{Test-And-Set Lock}

Рассмотрим следующий сценарий:
\only<1>{\begin{itemize}
  \item в ОС запущено несколько приложений, которые используют сеть;
  \item на компьютере установлена \emph{только одна} сетевая карта;
        \begin{itemize}
          \item когда сетевая карта заканчивает обрабатывать запрос - она
                генерирует прерывание (в современных сетевых устройствах все не
                так просто);
          \item обработчик прерывания берет следующий запрос из очереди;
          \item очередь запросов - общий ресурс;
	\end{itemize}
\end{itemize}}
\only<2->{\begin{enumerate}
  \item<2-> приложение хочет поставить запрос в очередь - оно захватывает
         spinlock;
  \item<3-> сетевая карта обработку запроса - генерирует прерывание;
  \item<4-> обработчик прерывания должен взять новый запрос из очереди - он
         пытается захватить spinlock;
  \item<5-> ... и все - мы в тупике ...
        \begin{itemize}
          \item приложение не может освободить spinlock, потому что прерывание
                прервало его работу;
          \item обработчик прерывания не может захватить spinlock, потому что он
                уже захвачен приложением;
        \end{itemize}
\end{enumerate}}
\end{frame}

\begin{frame}
\frametitle{Test-And-Set Lock}

\begin{itemize}
  \item<1-> Для Test-And-Set Lock-ов нужно выключать прерывания в начале lock и
        включать в конце unlock.
  \item<2-> Если прерывания единственный источник конкурентности (нет
        многоядерности), то достаточно просто выключить прерывания при входе в
        критическую секцию, чтобы гарантировать взаимное исключение;
\end{itemize}
\end{frame}

\begin{frame}[fragile]
\frametitle{Test-And-Set Lock}

Уменьшить нагрузку на системную шину можно следующим простым трюком:

\begin{lstlisting}
#define LOCK_FREE 0
#define LOCK_HELD 1

struct spinlock {
    int state;
};

void lock(struct spinlock *lock)
{
    disable_interrupts();
    while (1) {
        while (lock->state != LOCK_FREE) /* read only */
            barrier();

        if (cmpxchg(&lock->state, LOCK_FREE, LOCK_HELD) == LOCK_FREE)
            break;
    }
}

void unlock(struct spinlock *lock)
{
    xchg(&lock->state, LOCK_FREE);
    enable_interrupts();
}
\end{lstlisting}
\end{frame}

\begin{frame}
\frametitle{Test-And-Set Lock}

Текущая версия Test-And-Set Lock-а:
\begin{itemize}
  \item не приводит к deadlock-ам (тут трудно что-то улучшить);
  \item уменьшает нагрузку на системную шину;
        \begin{itemize}
          \item системная шина все еще довольно загружена - можно сделать лучше;
        \end{itemize}
  \item все еще не честная\onslide<2>{ - этим и займемся.};
\end{itemize}
\end{frame}

\begin{frame}[fragile]
\frametitle{Ticket Lock}

\begin{lstlisting}
struct ticketpair {
    uint16_t users;
    uint16_t ticket;
} __attribute__((packed));

struct spinlock {
    uint16_t users;
    uint16_t ticket;
}

void lock(struct spinlock *lock)
{
    disable_interrupts();
    const uint16_t ticket = atomic_add_and_get(&lock->users, 1);

    while (lock->ticket != ticket)
        barrier();
    smp_mb(); /* we don't use cmpxchg explicitly */
}

void unlock(struct spinlock *lock)
{
    smp_mb();
    atomic_add(&lock->ticket, 1);
    enable_interrupts();
}
\end{lstlisting}
\end{frame}

\begin{frame}
\frametitle{Ticket Lock}

\begin{itemize}
  \item Ticket Lock очень прост в реализации и достаточно эффективен;
  \item честность Ticket Lock-а не должна вызывать вопросов;
  \item реализация spinlock в Linux Kernel использует Ticket Lock.
\end{itemize}
\end{frame}

\begin{frame}
\frametitle{MCS Lock}

Ticket Lock, обычно, хорош на практике, зачем нам еще один Lock?
\begin{itemize}
  \item<1-> MCS Lock позволяет уменьшить нагрузку на системную шину до минимума
        не потеряв при этом в честности;
  \item<2-> простая и красивая идея (если есть что-то красивое в параллельном
        программировании, то это MCS Lock):
        \begin{itemize}
          \item авторы (John Mellor-Crummey and Michael L. Scott) получили приз
                Дейкстры в области распределенных вычислений;
          \item забавный факт - Дейкстра не первым получил премию Дейкстры...
        \end{itemize}
\end{itemize}
\end{frame}

\begin{frame}[fragile]
\frametitle{MCS Lock}
\framesubtitle{Опредления и реализация lock}
\begin{lstlisting}
struct mcs_lock {
    struct mcs_lock *next;
    int locked;
};

void lock(struct mcs_lock **lock, struct mcs_lock *self)
{
    struct mcs_lock *tail;

    self->next = NULL;
    self->locked = 1;

    tail = xchg(lock, self);

    if (!tail)
        return; /* there was no one in the queue */

    tail->next = self;
    barrier();
    while (self->locked);
        barrier();
    smp_mb();
}
\end{lstlisting}
\end{frame}

\begin{frame}[fragile]
\frametitle{MCS Lock}
\framesubtitle{Опредления и реализация unlock}
В отличие от других Lock-ов, unlock в MCS Lock менее тривиальная операция чем lock:
\begin{lstlisting}
void unlock(struct mcs_lock **lock, struct mcs_lock *self)
{
    struct mcs_lock *next = self->next;

    if (!next) {
         if (cmpxchg(lock, self, NULL) == node)
             return;

         while ((next = self->next) == NULL)
             barrier();
    }

    smp_mb();
    next->locked = 0;
}
\end{lstlisting}
\end{frame}

\begin{frame}
\frametitle{MCS Lock}

\begin{enumerate}
  \item MCS Lock определенно сложнее Ticket Lock и его интерфейс несколько
        отличается от привычного
        \begin{itemize}
          \item MCS Lock все еще достаточно прост;
          \item MCS Lock был первым подобным алгоритмом, но не последним - есть
                варианты с нормальным интерфейсом;
        \end{itemize}
  \item MCS Lock хранит всех претендентов в связном списке - честность
        вопросов не вызывает;
  \item нагрузка на системную шину минимальна:
        \begin{itemize}
          \item каждый ожидающий проверят свое и только свое поле locked;
        \end{itemize}
\end{enumerate}
\end{frame}
