\begin{frame}
\frametitle{Hazard Pointer}

\begin{itemize}
  \item Tagged Pointer - простая и достаточно эффективная техника, но не универсальная;
  \item Мы должны знать какого размера Tag-а нам точно достаточно:
    \begin{itemize}
      \item если поток заснул на долго, то Tag может переполнится и он этого не заметит;
      \item работа с большим Tag-ом может вообще не быть атомарной;
    \end{itemize}
  \item Возьмем Tagged Pointer на вооружение, но поищем альтернативу (тем более, что они есть);
\end{itemize}
\end{frame}

\begin{frame}
\frametitle{Hazard Pointer}

Hazard Pointer позволяет отметить указатель, как "небезопасный" и тем самым отложить его освобождение
\begin{itemize}
  \item оригинальная статья: "Hazard Pointers: Safe Memory Reclamation for Lock-Free Objects", Maged M. Michael;
  \item Hazard Pointer популярная универсальная техника, но мы рассмотрим только простую (и не универсальную) реализацию.
\end{itemize}
\end{frame}

\begin{frame}
\frametitle{Hazard Pointer}

Основная идея:
\begin{itemize}
  \item У каждого потока есть небольшой набор Hazard Pointer-ов, в которые он сохраняет указатели на используемые объекты (которые нельзя удалять);
  \item Писать в Hazard Pointer может только поток-владелец, но читать могут все;
  \item Перед тем как освобождать память проверим все Hazard Pointer-ы вех потоков.
\end{itemize}
\end{frame}

\begin{frame}[fragile]
\frametitle{Hazard Pointer}

Перед тем как рассмотреть реализацию, посмотрим как мы будем ее использовать, начнем с простого - реализация push:

\begin{lstlisting}
template <typename T>
void Stack<T>::push(T *value)
{
        StackNode *node = new StackNode(value);
        StackNode *expected;

        do {
                expected = this->head.load(std::memory_order_relaxed);
                node->next = expected;
        } while (!this->head.compare_exchange_strong(expected, node));
}
\end{lstlisting}
\end{frame}

\begin{frame}[fragile]
\frametitle{Hazard Pointer}

Тепрь реаизация pop:
\begin{lstlisting}
template <typename T>
T *Stack<T>::pop()
{
        HazardPtr<StackNode> hp(acquireRawHazardPtr(0));
        StackNode *node;
        StackNode *next;

        do {
                node = head.load();
                do {
                        if (!node)
                                return nullptr;
                        hp.set(node);
                } while ((node = head.load()) != hp.get());

                next = node->next;
        } while (!head.compare_exchange_strong(node, next));

        hp.set(nullptr);
        T *data = node->data;
        releasePtr(node);
        return data;
}
\end{lstlisting}
\end{frame}

\begin{frame}
\frametitle{Hazard Pointer}
Наша реализация будет очень простой потому что мы допустим несколько упрощений:
\begin{itemize}
  \item максимальное количество потоков нам известно заранее;
    \begin{itemize}
      \item полноценная реализация должна учитывать, что потоки могут создаваться и завершаться;
    \end{itemize}
  \item максимальное количество Hazard Pointer-ов на один поток нам известно заранее
    \begin{itemize}
      \item обычно потокам не нужно много Hazard Pointer-ов одновременно (3-4), поэтому это довольно слабое ограничение;
    \end{itemize}
  \item другими словами, мы можем статически алоцировать все Hazard Pointer-ы.
\end{itemize}
\end{frame}

\begin{frame}[fragile]
\frametitle{Hazard Pointer}
\begin{lstlisting}
static const std::size_t MAX_THREAD_COUNT = 1024;
static const std::size_t MAX_THREADPTR_COUNT = 16;
static const std::size_t MAX_HAZARD_COUNT = MAX_THREAD_COUNT * MAX_THREADPTR_COUNT;

typedef void * raw_ptr_t;
static std::atomic<raw_ptr_t> hazard[MAX_HAZARD_COUNT];

std::atomic<raw_ptr_t> *acquireRawHazardPtr(int idx)
{
    assert(idx < MAX_THREADPTR_COUNT);
    const int tid = getThreadId();
    return &hazard[tid * MAX_THREADPTR_COUNT + idx];
}
\end{lstlisting}

\begin{itemize}
  \item Hazard Pointer - глобальный атомарный указатель
    \begin{itemize}
      \item каждый поток может получить один из своих Hazard Pointer-ов по индексу (спасибо нашему упрощению);
      \item чтобы обезопасить указатель нужно сохранить его в атомарный указатель;
      \item когда вы закончите работать с указателем запишите туда 0;
    \end{itemize}
\end{itemize}
\end{frame}

\begin{frame}[fragile]
\frametitle{Hazard Pointer}

\begin{lstlisting}
void releaseRawPtr(raw_ptr_t ptr, std::function<void(raw_ptr_t)> deleter)
{
    static thread_local std::vector<RetireNode> retireList;
    static const size_t RETIRE_THRESHOLD = 10; // should be much bigger

    RetireNode node = { deleter, ptr };
    retireList.push_back(node);
    if (retireList.size() > RETIRE_THRESHOLD)
        collectGarbage(retireList);
}
\end{lstlisting}

\begin{itemize}
  \item Каждый поток поддерживает список ожидающих удаления элементов;
  \item Если список стал слишком большим - запускаем сборку "мусора";
\end{itemize}
\end{frame}

\begin{frame}[fragile]
\frametitle{Hazard Pointer}

\begin{lstlisting}
static void collectGarbage(std::vector<RetireNode> &retireList)
{
        std::set<raw_ptr_t> plist;
        for (std::size_t i = 0; i != MAX_HAZARD_COUNT; ++i) {
                raw_ptr_t ptr = hazard[i].load();

                if (ptr != nullptr)
                        plist.insert(ptr);
        }

        std::vector<RetireNode> retire;
        retire.swap(retireList);
        for (std::size_t i = 0; i != retire.size(); ++i) {
                if (plist.count(retire[i].ptr))
                        retireList.push_back(retire[i]);
                else
                        retire[i].deleter(retire[i].ptr);
        }
}
\end{lstlisting}

\begin{itemize}
  \item Для каждого элемента в списке проверим, защищен ли он Hazard Pointer-ом:
    \begin{itemize}
      \item если да, то вернем указатель назад в список ожидания;
      \item если нет, то его можно безопасно удалять;
    \end{itemize}
\end{itemize}
\end{frame}

\begin{frame}
\frametitle{Финальные замечания}

\begin{itemize}
  \item Lockless алгоритмы не редко трудны для реализации и понимания
    \begin{itemize}
      \item используйте Lock-и по умолчанию, и Lockless, когда Lock-и стали проблемой;
    \end{itemize}
  \item Lockless алгоритмы не производительнее и не масштабируются лучше чем Lock-based:
    \begin{itemize}
      \item Lockless алгоритмы предоставляют гарантии прогресса, а не производительности;
      \item на производительности и масштабируемости в первую очередь сказывается конкуренция за общие ресурсы;
    \end{itemize}
\end{itemize}
\end{frame}
